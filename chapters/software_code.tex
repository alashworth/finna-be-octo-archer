\chapter{API Interface}

Mains-powered boards remain on at all times, and can be accessed and queried via a web browser. This appendix lists and describes the resources used as of the publication date of this thesis.

\section*{Possible Node Responses}
\begin{itemize}
\item 200 OK - The request was successful (some API calls may return 201 instead)
\item 400 Bad Request - The request could not be understood or was missing required parameters.
\item 404 Not Found - Resource not found.
\item 405 Method Not Allowed - Requested method is not supported for the specified resource.
\end{itemize}

When querying for a sensor result, it will return in JSON format. Example:

\begin{lstlisting}
GET /sensor/temperature/{id} HTTP/1.1

200 (OK)
Content-Type: application/json

{
	"result": 22.5,
	"timestamp": "2014-03-15T19:20:30Z"
}
\end{lstlisting}

\section*{Currently Supported Commands}

For the board described in this document, the temperature sensor ID is 1, the humidity sensor ID is also 1, and the light sensor has ID 2.

\begin{table}[h]
\begin{tabular}{@{}lll@{}}
\toprule
\textbf{GET} & \multicolumn{2}{l}{/sensor/{id}/temperature} \\ \midrule
\multicolumn{3}{l}{Return the current temperature in Celsius.} \\
\multicolumn{3}{c}{\textbf{Parameter}} \\
\textbf{Name} & \textbf{Description} & \textbf{Details} \\
id & the id number of the sensor & number, required \\ \bottomrule
\end{tabular}
\end{table}
\begin{table}[h]
\begin{tabular}{@{}lll@{}}
\toprule
\textbf{GET} & \multicolumn{2}{l}{/sensor/{id}/humidity} \\ \midrule
\multicolumn{3}{l}{Return the current relative humidity as a percentage} \\
\multicolumn{3}{c}{\textbf{Parameter}} \\
\textbf{Name} & \textbf{Description} & \textbf{Details} \\
id & the id number of the sensor & number, required \\ \bottomrule
\end{tabular}
\end{table}
\begin{table}[h]
\begin{tabular}{@{}lll@{}}
\toprule
\textbf{GET} & \multicolumn{2}{l}{/sensor/{id}/light} \\ \midrule
\multicolumn{3}{l}{Return the current luminance in lux.} \\
\multicolumn{3}{c}{\textbf{Parameter}} \\
\textbf{Name} & \textbf{Description} & \textbf{Details} \\
id & the id number of the sensor & number, required \\ \bottomrule
\end{tabular}
\end{table}

\section*{Code Listing}

The code was written using the QP Framework QM State-Machine modeling tool. The full source to this thesis can be found at \url{https://github.com/alashworth/cc3200-devl}.