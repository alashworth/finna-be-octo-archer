\chapter{Conclusion}

The results described within have wide applicability to a wide variety of wireless devices. Compact size, easy-to-use software, and high energy efficiency are universally desired properties of most mobile devices today. RF transmission and receive can consume order of magnitudes more power than the signal processing for the workloads involved in environmental sensing due to the low computational complexity involved and the low current consumption of today's modern integrated sensors and microcontrollers. Techniques to reduce data transmission can drastically increase battery lifetimes. Similarly, 

Given the typical duty cycle and mode of operation of a wireless node, the wireless channel remains the dominant 

\section{Future Work}

There remains a great deal of work to do in terms of firmware. As seen in the API appendix, currently only three querying commands are supported, and only for mains-powered boards. Battery-powered devices push data to the server, instead of the server pulling data from the sensor. In the future, it would be useful if battery-powered devices were accessed in the same way as mains-powered devices. This requires some method to emulate always-on behavior, as battery-powered devices may be asleep and unable to respond to a synchronously. One way to do this would be to have a mains-powered device act as a caching proxy for requests to battery-powered nodes. In fact, this is an area of open research\cite{lu2011application}. Additional and helpful software features would be to support automatic sensor discovery and configuration.