\chapter{Abstract}

The ecoMOD project at UVA is an effort from the architectural school in designing low-cost, energy-efficient housing. In order to validate architectural design choices, the buildings must be instrumented post-construction. Building environment monitoring can give key feedback on energy and resource consumption. This work explores the design of a small, energy efficient, wireless sensing platform. Since sometimes, residential sensors cannot be co-located with power access, low-power design must be a factor in node design. The system's power usage is analyzed and design choices are made to minimize battery consumption. By building a functional prototype, system feasibility in monitoring buildings is demonstrated.

\listoftodos